\documentclass{article}

\usepackage{tabularx}
\usepackage{caption}
\usepackage{wrapfig}
\usepackage{graphicx}
\usepackage{hyperref}

\title{PHYC 152L}
\author{Jonathan Gross}
\date{Fall 2012}

\begin{document}
\maketitle
\noindent Email: \href{mailto:jagross@unm.edu}{jagross@unm.edu} \\

Hello - I am here to help you do your labs and learn some physics that will
help you in your lecture class and future courses. Always feel free to ask
questions or make suggestions.  Also, feel free to talk to the Lab Director in
room RH 113, phone 277-2751. Welcome to Physics 152L!

An electronic version of this document is available online at
\url{http://unm.edu/~jagross/courses/phyc152l}.

\section*{Description}
Physics 152L is the laboratory associated with the 152 lecture covering topics
in electricity, magnetism, geometrical optics and wave optics. Hands on
experiments involving data collection and analysis give students a better
conceptual framework for understanding physics. Lab experiments mirror and
enhance lecture topics. \emph{(Pre- or co-requisite: Physics 152)}

\section*{Objectives}
This course serves to reinforce concepts presented in lecture and to
familiarize students with various experimental techniques. Lab students will:
\begin{itemize}
\item Read and interpret procedural instructions
\item Gather and analyze data using electronic and optical devices
\item Observe electrical, magnetic and optical phenomena
\item Relate observed phenomena to mathematical and physical models
\item Use basic laboratory equipment (e.g., timer, balance, rods, clamps, etc.)
\end{itemize}

\section*{Attendance}
Lab attendance is very much mandatory.
\begin{description}
\item[Scheduled absences:] If you cannot make it to your usual section,
attending a different lab section is highly desirable.  Contact me BEFORE the
class meeting to make sure there is room.  If you wish to attend a lab that is
not taught by myself, contact BOTH TAs before the appropriate times.
\item[Unscheduled absences:] Unscheduled absences WITH A VAILD EXCUSE must be
made up in the following week. Contact me to arrange a mutually convenient
time.
\item[Unexcused absences:] One unexcused absence will cause your final grade to
drop one letter grade.  Two unexcused absences will cause you to fail the
class.
\end{description}

\section*{Materials}
Lab workbooks are available at the Bookstore.  Every student is required to
purchase one before the second week of lab. No copies will be provided.

\section*{Preparation}
Glance at the lab before coming to class. The order of the labs is
approximately the order in which they appear in the laboratory workbook, but
check the posted schedule at \url{http://panda.unm.edu/regener/lab/}.

\section*{Students with Disabilities}
\begin{wrapfigure}[4]{l}{0.15\textwidth}
  \vspace{-0.04\textwidth}
  \centering
  \includegraphics[width=0.14\textwidth]{Handicap.pdf}
\end{wrapfigure}
Qualified students with disabilities needing appropriate academic adjustments
should contact me as soon as possible to ensure your needs are met. Handouts
are available in alternative accessible formats upon request.

\section*{Grading}
Each week, you will receive a score with three components: one for attendance,
one for attitude and participation, and one for the physics and other details
in the lab report. Score for everything well done: 2(attendance) +
2(participation) + 4(lab report) = 8. A score of 8 corresponds to a letter
grade of A. Your semester grade will be based on the weekly scores and on
quizzes which may be given at announced times in the semester.

Below are guidelines for the weekly scores:

\begin{table}[htbp]
\caption*{Attendance}
\begin{tabularx}{\linewidth}{|r|X|}
\hline
\emph{Grade} & \emph{Situation} \\ \hline
0 & Not present in your section. Attendance in another section not verified by
    TA. \\ \hline
2 & Attendance \emph{verified by TA}. \\ \hline
\end{tabularx}
\label{}
\end{table}

\begin{table}[htbp]
\caption*{Attitude/Participation}
\begin{tabularx}{\linewidth}{|r|l|X|}
\hline
\multicolumn{1}{|l|}{\emph{Grade}} & \emph{Situation} & \emph{Example Behavior} \\ \hline
0 & No participation & Reading newspaper, doing homework, antagonizing lab
                       partners, etc. \\ \hline
1 & Passive participation & Merely recording data, not helping with set up, not
                            participating in discussions, copying partners,
                            etc. \\ \hline
2 & Active participation & Helping with set up, participating in data taking,
                           asking questions, participating in discussions, etc.
                           \\ \hline
\end{tabularx}
\label{}
\end{table}

\begin{table}[htbp]
\caption*{Lab Report}
\begin{tabularx}{\linewidth}{|r|l|X|}
\hline
\multicolumn{1}{|l|}{\emph{Grade}} & \emph{Quality} & \emph{Example} \\ \hline
0 & No report & Blank paper \\ \hline
1 & Unacceptable & Student turned in report but it was unacceptably poor.
                   Significant errors, unanswered questions, missing data, etc.
                   \\ \hline
2 & Acceptable & Report is OK, but there are small errors or missing entries
                 (minimal). \\ \hline
3 & Good & Report has only minor error(s). \\ \hline
4 & Standard & The report is everything I would expect. All entries are
               complete, all questions meaningfully answered, data record
               including graphs is clear and correct, all calculations and
               units are correct. The report is organized and legible.
               \\ \hline
\end{tabularx}
\label{}
\end{table}
\end{document}
